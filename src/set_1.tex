\documentclass{amsart}

\usepackage{amsthm}
\usepackage{amsmath}
\usepackage{amsfonts}
\usepackage{amssymb}
\usepackage{fullpage}
\usepackage{xcolor}
\usepackage{textcomp}
\usepackage{graphicx}
\usepackage{mathtools}
\usepackage{algorithm}
\usepackage{algpseudocode}
\newtheorem*{theirtheorem}{Theorem}
\newtheorem*{theirproposition}{Proposition}


\theoremstyle{plain}
\newtheorem*{theorem}{\textbf{Theorem}}
\newtheorem*{lemma}{\textbf{Lemma}}
\newtheorem*{corollary}{\textbf{Corollary}}
\newtheorem*{proposition}{\textbf{Proposition}}
\newtheorem*{claim}{\textbf{Claim}}
\newtheorem*{conjecture}{\textbf{Conjecture}}

\theoremstyle{definition}
\newtheorem*{rk}{\textbf{Remark}}

\newcommand{\Summ}[1]{\underset{#1}{\sum}}
\newcommand{\sti}[2]{\left\{\begin{array}{c} #1 \\ #2 \end{array}\right\}}

\newcommand{\diam}{\emph{diam}}
\newcommand{\conv}{\mbox{Conv}}
\newcommand{\C}{\mathcal {C}}
\newcommand{\R}{\mathbb{R}}
\newcommand{\Z}{\mathbb{Z}}
\newcommand{\N}{\mathbb{N}}
\newcommand{\F}{\mathbb{F}}

\newcommand{\B}{\mathcal{B}}
\newcommand{\A}{\mathcal{A}}
\newcommand{\G}{\mathcal{G}}
\newcommand{\D}{\mathcal{D}}

\newcommand{\np}{\textbf{NP}}
\newcommand{\dtime}{\textbf{DTIME}}
\newcommand{\sol}{\textbf{Solution:\,}}

\newcommand{\ov}[1]{\overline{#1}}

\newcommand{\nn}{\nonumber}

\thispagestyle{empty}

\usepackage{enumitem}

\usepackage{parskip}


\begin{document}

    {\Large MIRI, MAMME, MPAL -- Computational Complexity -- Homework 1}

    \vspace{0.5cm}

    \hrule

    \vspace{0.5cm}

    \begin{enumerate}[label=\textbf{Exercise \arabic*:}, leftmargin=0cm, labelwidth=-0.2cm, align=left]

        \item
            Prove that \verb|DIRECTED DOMINATING SET| is \np-Hard through a series of Karp reductions
            starting at \verb|3SAT|.

            \sol I will first introduce some notation.
            Let $F$ be a boolean formula with $N$ variables and $M$ CNF
            clauses with $3$ literals each.u,  v
            I will denote the variables as $x_1, x_2, \ldots, x_N$ and the clauses as $c_1, c_2, \ldots, c_M$.
            For convenience, I will enumerate all possible literals as $l_j = x_j, l_{N+j} = \overline{x_j}$
            (there are $2N$ of them).
            I will denote
            $c_i = \left( l_{j_{i, 1}}, l_{j_{i, 2}}, l_{j_{i, 3}} \right) =
            \left( l^i_1, l^i_2, l^i_3 \right)
            \in \{l_1, \, \dots \, , l_{2N}\}^3$.

            I will now construct a directed graph $D = (V, A)$ and $k \in \N$ such that
            $D$ contains a dominating set of size at most $k$ if and only if $F$ is satisfiable.
            Furthermore, the construction of the graph will clearly be polynomial in time,
            thus providing the Karp reduction we need directly.
            This construction is my own work, although I can't garantee that
            a similar construction hasn't been used before, as my literature review was not exhaustive,
            and it seems like a natural approach.


            First, I define the vertices $V$ of $D$ as:
            \begin{itemize}
                \item A vertex $L_j$ for each literal $l_j$
                ($2N$ in total, which can be created in linear time by scanning $F$).
                For convenience, I will denote $L^i_u \coloneqq L_{j_{i, u}}$.

                \item A vertex $C_i$ for each clause $c_j$ ($M$ in total, which similarly can be created in linear time).

            \end{itemize}

            Next, I define the edges $A$ of $D$ as:
            \begin{itemize}
                \item $\left( L^i_u, C_i \right)$ for $1 \leq u \leq 3,\, 1 \leq i \leq M$
                (each literal points to the clauses it appears in, which we can construct in linear time).

                \item $\left( L_s, L_{s+N} \right)$ and $\left( L_{s+N}, L_{s} \right)$ for $1 \leq s \leq N$
                (each literal points to its negation, which we can construct in linear time).

            \end{itemize}

            Finally, I define $k = N$.
            It remains to be proven that $D$ has a dominating set of size $k \iff F$ is satisfiable:

            % prove both implications separately

            \begin{itemize}

                \item[$\Leftarrow\text{)}$] Suppose we have an assignment $x_s = B_s \in \{\text{True}, \text{False}\}$
                that satisfies $F$.
                I will show that the set $S \coloneqq \{l_s | B_s = \text{True}\} \cup \{l_{s+N} | B_s = \text{False}\}$,
                which has size $N=k$, is dominating:
                \begin{itemize}
                    \item All $L_j$ are either in $S$ or pointed to by $ \overline{L_j} \coloneqq L_{j \pm N} \in S$.
                    \item All $C_i$ are pointed to by their literals, at least one of which is in $S$.
                \end{itemize}

                \item[$\Rightarrow\text{)}$] Suppose there is a dominating set $S$ of size at most $k$.
                for each variable $x_s$, $L_s$ must either be in $S$ or pointed to by an element of $S$
                (that is, one of $L_s, L_{s+N}$ is in $S$). In fact, because there are $N=k$ variables,
                \emph{exactly} one of them is in $S$ (otherwise $S$ would have more than $k$ elements).
                Furthermore, $S$ only contains vertices of the form $L_j$ (and not $C_i$).

                This means that a variable assignment $x_s = B_s$ where $B_s = \text{True}$
                if $L_s \in S$ and $B_s = \text{False}$ if $L_{s+N} \in S$ can be defined.
                To show that this assignment satisfies $F$, note that for each clause $c_i$, there is
                a literal $l_j$ such that $L_j$ points to $C_i$ and $L_j \in S$.
                If $j \leq N$, this means we have assigned $x_j = \text{True}$ and $x_j = l^i_u$ for $1 \leq i \leq 3$,
                satisfying the clause.
                Otherwise, we have assigned $x_{j-N} = \text{False}$ and $\overline{x_{j-N}} = l^i_u$ for $1 \leq i \leq 3$,
                satisfying the clause as well.

            \end{itemize}

            \begin{rk}
                This construction works just as well for arbitrary \verb|SAT| instances, not just \verb|3SAT|.
            \end{rk}

        \newpage

        \item \textcolor{white} a

             \begin{enumerate}[label=\alph*)]

            \item \label{itm:2a}
            Show that every tournament with $n$ nodes has a dominating set of size at most $\lceil \log n \rceil$.

            \sol I will prove this by induction on $n$, assuming that $\log$ is the base $2$ logarithm.
            Let $T = (V, A)$ be a tournament with $n$ nodes.
            Note that the statement is not true for $n \in \{0, 1\}$.
            If $n=2$, the graph consists of two nodes $a, b$ and an edge $(a, b)$: a dominating set is $S = \{a\}$
            of size $1 = \lceil \log 2 \rceil$.

            Suppose now that $n > 2$ and that the result holds for all tournaments with less than $n$ nodes.
            we define the out-degree of a node $v$ as $d_{D}(v) = |\{u | (u,  v) \in A\}|$.
            because each edge has exactly one source, we have
            \[\
                \sum_{v \in V} d_{D}(v) = |A| = \frac{n(n-1)}{2}
            \]
            By the pigeonhole principle, there must be a node $v$ with $d \coloneqq d_{D}(v) \geq \frac{n-1}{2}$.
            Let $T'$ be the tournament obtained by removing from $T$ $v$ and all vertices it points to.
            It has
            \[n' \coloneqq n - d - 1 \leq n - \frac{n-1}{2} - 1 = \frac{n-1}{2}\]
            vertices.
            By the induction hypothesis,
            $T'$ has a dominating set $S'$ of size at most $\left \lceil \log n' \right \rceil$.
            Let
            \[
                S = S' \cup \{v\}
            \]
            Clearly, by construction, $S$ is a dominating set of $T$.
            Furthermore,
            \[
                |S| = |S'| + 1 \leq
                \left \lceil \log n' \right \rceil + 1 \leq
                \left \lceil \log \left( \frac{n-1}{2} \right)\ \right \rceil + 1 =
                \left \lceil \log (n-1) - 1 \right \rceil + 1 =
                \left \lceil \log (n - 1) \right \rceil \leq
                \left \lceil \log n \right \rceil
            \]

            The only case where the induction hypotesis cannot be applied is when $T'$ has less
            than $2$ nodes.
            In that case a dominating set of size at most $1$
            can be clearly constructed by taking the only node in $T'$
            (or the empty set if $T'$ has no nodes).
            In conclusion $|S| \leq 1 + 1 = 2 \leq \lceil \log n \rceil$.


            \item Prove that if \verb|TOURNAMENT DOMINATING SET| is
            $\np$-Complete, then $\np \subseteq \dtime(n^{O(\log n)})$.

            \sol It is sufficient to show that there is an algorithm that decides \verb|TOURNAMENT DOMINATING SET|
            in time $n^{O(\log n)}$.
            I describe it below:

            \begin{algorithm}
                \caption{algorithm for TOURNAMENT DOMINATING SET}
                \label{alg:alg}
                \begin{algorithmic}[1]
                    \Function{ExistsDominatingSetInTournament}{$T = (V, A), k \geq 0$}
                        \If {$|V| = 0$}
                            \State \Return \textbf{true}
                        \EndIf

                        \If {$k = 0$}
                            \State \Return \textbf{false}
                        \EndIf

                        \If {$|V| = 1$}
                            \State \Return \textbf{true}
                        \EndIf

                        \If {$k \geq \lceil \log |V| \rceil$}
                            \State \Return \textbf{true}
                        \EndIf

                        \ForAll{$S \in \binom{V}{k}$}
                            \If {$S$ is a dominating set}
                                \State \Return \textbf{true}
                            \EndIf

                        \EndFor

                        \State \Return \textbf{false}

                    \EndFunction
                \end{algorithmic}
            \end{algorithm}

            The algorithm~\ref{alg:alg} is a brute-force algorithm that tries all possible sets of size $k$.
            By~\ref{itm:2a}, we know that if $k \geq \lceil \log |V| \rceil$, there is a dominating set of size at most $k$,
            so the algorithm will return \textbf{true} which is correct.
            Because the loop only runs when $k < \lceil \log |V| \rceil$,
            enumerating all sets can be done in time $O(n^k) = n^{O(\log n)}$~\cite{reingold1977combinatorial}.
            the rest of the algorithm is clearly polynomial in time.
            For appropriate polynomials $p, q$, the algorithm
            runs in time $p(n) + q(n) n^{O(\log n)} = n^{O(\log n)}$.

            \end{enumerate}

    \end{enumerate}

    \bibliography{main}
    \bibliographystyle{plain}



\end{document}
