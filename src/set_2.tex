\documentclass{amsart}

\usepackage{amsthm}
\usepackage{amsmath}
\usepackage{amsfonts}
\usepackage{amssymb}
\usepackage{fullpage}
\usepackage{xcolor}
\usepackage{textcomp}
\usepackage{graphicx}
\usepackage{mathtools}
\usepackage{algorithm}
\usepackage{algpseudocode}
\newtheorem*{theirtheorem}{Theorem}
\newtheorem*{theirproposition}{Proposition}


\theoremstyle{plain}
\newtheorem*{theorem}{\textbf{Theorem}}
\newtheorem*{lemma}{\textbf{Lemma}}
\newtheorem*{corollary}{\textbf{Corollary}}
\newtheorem*{proposition}{\textbf{Proposition}}
\newtheorem*{claim}{\textbf{Claim}}
\newtheorem*{conjecture}{\textbf{Conjecture}}

\theoremstyle{definition}
\newtheorem*{rk}{\textbf{Remark}}

\newcommand{\Summ}[1]{\underset{#1}{\sum}}
\newcommand{\sti}[2]{\left\{\begin{array}{c} #1 \\ #2 \end{array}\right\}}

\newcommand{\diam}{\emph{diam}}
\newcommand{\conv}{\mbox{Conv}}
\newcommand{\C}{\mathcal {C}}
\newcommand{\R}{\mathbb{R}}
\newcommand{\Z}{\mathbb{Z}}
\newcommand{\N}{\mathbb{N}}
\newcommand{\F}{\mathbb{F}}

\newcommand{\B}{\mathcal{B}}
\newcommand{\A}{\mathcal{A}}
\newcommand{\G}{\mathcal{G}}
\newcommand{\D}{\mathcal{D}}

\newcommand{\np}{\textbf{NP}}
\newcommand{\p}{\textbf{P}}
\newcommand{\conp}{\textbf{co\,-NP}}
\newcommand{\ph}{\textbf{PH}}
\newcommand{\dtime}{\textbf{DTIME}}
\newcommand{\sol}{\textbf{Solution:\,}}

\newcommand{\ov}[1]{\overline{#1}}

\newcommand{\nn}{\nonumber}

\thispagestyle{empty}

\usepackage{enumitem}

\usepackage{parskip}


\begin{document}

    {\Large MIRI, MAMME, MPAL -- Computational Complexity -- Homework 2}

    \vspace{0.5cm}

    \hrule

    \vspace{0.5cm}

    \begin{enumerate}[label=\textbf{Exercise \arabic*:}, leftmargin=0cm, labelwidth=-0.2cm, align=left]

        \item
            Argue that if $\Delta_k^\p = \Sigma_k^\p$, then $\Delta_k^\p=$\ph\@.

            \sol My plan is to prove that if $\Delta_k^\p = \Sigma_k^\p$, then $\Delta_k^\p=\Pi_k^\p$.
            This implies that $\Sigma_k^\p = \Pi_k^\p$, and we have seen in class that
            this implies $\Sigma_k^\p = \Pi_k^\p = \ph$.

            For this, recall the recursive definition of these classes:
            \begin{align*}
                \Delta_k^\p &= \p^{\Sigma_i^\p}. \\
                \Sigma_k^\p &= \np^{\Sigma_i^\p}. \\
                \Pi_k^\p &= \conp^{\Sigma_i^\p}.
            \end{align*}

            The inclusions $\Delta_k^\p \subseteq \Sigma_k^\p$ and $\Delta_k^\p \subseteq \Pi_k^\p$ clear
            from the definitions and have been discussed in class.
            Therefore, it is enough to prove

            \begin{equation}
                \Sigma_k^\p \subseteq \Delta_k^\p \implies \Pi_k^\p \subseteq \Delta_k^\p.\label{eq:want}
            \end{equation}

            Suppose that $\Sigma_k^\p \subseteq \Delta_k^\p$ and let $X \in \Pi_k^\p$ be a language.
            Then, $\ov{X} \in \Sigma_k^\p \subseteq \Delta_k^\p$.
            However, $\Delta_k^\p$ is closed under compliment, because its languages can be decided
            deterministically in polynomial time with a $\Sigma_i^\p$ oracle
            (in particular, the output bit can just be flipped).
            More formally, let $M$ be a DTM that decides $\ov{X}$ with access to a $\Sigma_i^\p$ oracle.
            Then, a DTM $M'$ that decides $X$ with access to the same oracle can be constructed:
            Run $M$ on the input and output the opposite of the answer given by $M$.
            This implies that $X \in \Delta_k^\p$, proving~\eqref{eq:want}.




    \end{enumerate}




\end{document}
