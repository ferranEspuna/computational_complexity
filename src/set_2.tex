\documentclass{amsart}

\usepackage{amsthm}
\usepackage{amsmath}
\usepackage{amsfonts}
\usepackage{amssymb}
\usepackage{fullpage}
\usepackage{xcolor}
\usepackage{textcomp}
\usepackage{graphicx}
\usepackage{mathtools}
\usepackage{algorithm}
\usepackage{algpseudocode}
\newtheorem*{theirtheorem}{Theorem}
\newtheorem*{theirproposition}{Proposition}


\theoremstyle{plain}
\newtheorem*{theorem}{\textbf{Theorem}}
\newtheorem*{lemma}{\textbf{Lemma}}
\newtheorem*{corollary}{\textbf{Corollary}}
\newtheorem*{proposition}{\textbf{Proposition}}
\newtheorem*{claim}{\textbf{Claim}}
\newtheorem*{conjecture}{\textbf{Conjecture}}

\theoremstyle{definition}
\newtheorem*{rk}{\textbf{Remark}}

\newcommand{\Summ}[1]{\underset{#1}{\sum}}
\newcommand{\sti}[2]{\left\{\begin{array}{c} #1 \\ #2 \end{array}\right\}}

\newcommand{\diam}{\emph{diam}}
\newcommand{\conv}{\mbox{Conv}}
\newcommand{\C}{\mathcal {C}}
\newcommand{\R}{\mathbb{R}}
\newcommand{\Z}{\mathbb{Z}}
\newcommand{\N}{\mathbb{N}}
\newcommand{\F}{\mathbb{F}}

\newcommand{\B}{\mathcal{B}}
\newcommand{\A}{\mathcal{A}}
\newcommand{\G}{\mathcal{G}}
\newcommand{\D}{\mathcal{D}}

\newcommand{\np}{\textbf{NP}}
\newcommand{\p}{\textbf{P}}
\newcommand{\nl}{\textbf{NL}}
\newcommand{\conp}{\textbf{co\,-NP}}
\newcommand{\ph}{\textbf{PH}}
\newcommand{\dtime}{\textbf{DTIME}}
\newcommand{\sol}{\textbf{Solution:\,}}

\newcommand{\ov}[1]{\overline{#1}}

\newcommand{\nn}{\nonumber}

\thispagestyle{empty}

\usepackage{enumitem}

\usepackage{parskip}


\begin{document}

    {\Large MIRI, MAMME, MPAL -- Computational Complexity -- Homework 2}

    \vspace{0.5cm}

    \hrule

    \vspace{0.5cm}

    \begin{enumerate}[label=\textbf{Exercise \arabic*:}, leftmargin=0cm, labelwidth=-0.2cm, align=left]

        \item
            Argue that if $\Delta_k^\p = \Sigma_k^\p$, then $\Delta_k^\p=$\ph\@.

            \sol My plan is to prove that if $\Delta_k^\p = \Sigma_k^\p$, then $\Delta_k^\p=\Pi_k^\p$.
            This implies that $\Sigma_k^\p = \Pi_k^\p$, and we have seen in class that
            this implies $\Sigma_k^\p = \Pi_k^\p = \ph$.

            For this, recall the recursive definition of these classes:
            \begin{align*}
                \Delta_k^\p &= \p^{\Sigma_{k-1}^\p}. \\
                \Sigma_k^\p &= \np^{\Sigma_{k-1}^\p}. \\
                \Pi_k^\p &= \conp^{\Sigma_{k-1}^\p}.
            \end{align*}

            The inclusions $\Delta_k^\p \subseteq \Sigma_k^\p$ and $\Delta_k^\p \subseteq \Pi_k^\p$ clear
            from the definitions and have been discussed in class.
            Therefore, it is enough to prove

            \begin{equation}
                \Sigma_k^\p \subseteq \Delta_k^\p \implies \Pi_k^\p \subseteq \Delta_k^\p.\label{eq:want}
            \end{equation}

            Suppose that $\Sigma_k^\p \subseteq \Delta_k^\p$ and let $X \in \Pi_k^\p$ be a language.
            Then, $\ov{X} \in \Sigma_k^\p \subseteq \Delta_k^\p$.
            However, $\Delta_k^\p$ is closed under complement, because its languages can be decided
            deterministically in polynomial time with a $\Sigma_{k-1}^\p$ oracle
            (in particular, the output bit can just be flipped).
            More formally, let $M$ be a DTM that decides $\ov{X}$ with access to a $\Sigma_{k-1}^\p$ oracle.
            Then, a DTM $M'$ that decides $X$ with access to the same oracle can be constructed:
            Run $M$ on the input and output the opposite of the answer given by $M$.
            This implies that $X \in \Delta_k^\p$, proving~\eqref{eq:want}.

        \item
            Let $A, B \subseteq \{0, 1\}^*$ be two languages.
            Show that if $A \leq_m^l B$ and $B \in \nl$, then $A \in \nl$.

            \sol Recall that $A \leq_m^l B$ means that there is a log-space computable function $f$
            (say, by a deterministic Turing machine $D$) such that
            \begin{equation}
                x \in A \iff f(x) \in B.
                \label{eq:logspace}
            \end{equation}

            On the other hand, $B \in \nl$ means that there is a log-space non-deterministic Turing machine $M$
            such that $M$ accepts $x$ if and only if $x \in B$.
            Instead of using certificates, which is more cumbersome,
            I use the formalism that $M$ has two transition functions $\delta_0$ and $\delta_1$ and accepts if and only if
            any choices of $\delta_0$ and $\delta_1$ for all transitions leads to the accepting state.

            Naïvely, a non-deterministic turing machine $N$ can be constructed that computes $f(x)$ by running $D$ and then runs $M$ on the output.
            However, this machine would not necessarily run in log-space, because it would need to write the output of $f$ on one of its
            work tapes in order for $M$ to read it.
            Instead, use a trick introduced in class: I construct the $k$th bit of $f(x)$ whenever it is needed by $M$.
            To achieve this, in addition to the input and work tapes of $D$, and the output and work tapes of $M$, I need:
            \begin{itemize}
                \item A work tape $K$ of which only one bit is used, which corresponds to the $k$th bit of $f(x)$.
                This serves as the output tape of $D$ and the input tape of $M$.
                The head associated with this tape never moves.

                \item A work tape $MC$ to store the position of the read head of $M$, and increment it or decrement it
                whenever the program of $M$ moves the read head.

                \item A work tape $DC$ to store the position of the write head of $D$, and increment it or decrement it
                whenever the program of $D$ moves the write head.
            \end{itemize}

            More precisely, the machine works as follows:

            \begin{algorithm}[H]
            \caption{Non-deterministic Turing Machine $N$ that decides $A$}
            \label{alg:kpartite}
            \begin{algorithmic}[1]
                \Require $x$
                \State $MC \gets 0$
                \For{step $S$ of a run of $M$ from its initial configuration} \label{line:S}
                    \If{$S$ reads from the input}
                        \State $DC \gets 0$
                        \For{step $T$ of a run of $D$ on the input $x$ from its initial configuration}
                            \If{$T$ writes bit $b$ to the output tape and $DC = MC$}
                                \State Write $b$ in $K$
                            \EndIf
                            \State Perform step $T$, without moving $N$'s write head or writing to its output tape
                            \If{$T$ moves the write head to the left}
                                \State $DC \gets DC - 1$
                            \ElsIf{$T$ moves the write head to the right}
                                \State $DC \gets DC + 1$
                            \EndIf
                        \EndFor
                    \EndIf
                    \State Perform step $S$, reading from $K$ instead of the input tape, and never moving $N$'s read head

                    \If{$S$ moves the read head to the left}
                        \State $MC \gets MC - 1$

                    \ElsIf{$S$ moves the read head to the right}
                        \State $MC \gets MC + 1$

                    \EndIf
                \EndFor
                
                    \end{algorithmic}
        \end{algorithm}

        Note that the machine $N$ is a (non-deterministic, because the steps of the machine
        $M$ used are non-deterministic) Turing machine that completely emulates the machine $M$
        on the input $f(x)$.
        Therefore, it accepts $x$ if and only if $M$ accepts $f(x)$.
        That is, if and only if $x \in A$.
        Furthermore, it only uses as much space as $M$ and $D$ together, plus
        the space used in the tapes $MC$, $DC$ and $K$.
        However, $K$ is only one bit, and $MC$ and $DC$ can count only up to $|f(x)|$, which is polynomial in $|x|$,
        because $D$ is a log-space machine.
        Therefore, the number of bits used by $N$ is logarithmic in $|x|$.
        This concludes the proof that $A \in \nl$.

        The interplay between the non-determinism of $M$ and that of $N$ is kind of ``hidden'' by the formalism
         of choice between two transition functions (which is performed every time a step $S$ is selected
        in line~\ref{line:S}). To do the same using the certificates formalism, I would model both $N$ and $M$ as
        deterministic verifiers. $N$ would be a  $DTM$ taking
        $\left< x, u \right>$ as input, where $u$ is a certificate of size polynomial in $|f(x)|$
        (which is polynomial in $|x|$).
        Then, I would change $D$ to $D'$, where $D'$
        is a log-space $DTM$ that computes $f'(\left< x, u \right>) = \left< f(x), u \right>$
        (this is possible because $D$ is a log-space machine and copying $u$ from the input to the output
        tape has no effect on the space used).
        Then, $N$ accepts $\left< x, u \right>$ if and only if $M$ accepts $\left< f(x), u \right>$.
        Therefore, the $NDTM$ associated with $N$ accepts $x$ if and only if
        the $NDTM$ associated with $M$ accepts $f(x)$.
        That is, if and only if $x \in A$.



    \end{enumerate}

\end{document}
