\documentclass{amsart}

\usepackage{amsthm}
\usepackage{amsmath}
\usepackage{amsfonts}
\usepackage{amssymb}
\usepackage{fullpage}
\usepackage{xcolor}
\usepackage{textcomp}
\usepackage{graphicx}
\usepackage{mathtools}
\usepackage{algorithm}
\usepackage{algpseudocode}
\newtheorem*{theirtheorem}{Theorem}
\newtheorem*{theirproposition}{Proposition}


\theoremstyle{plain}
\newtheorem*{theorem}{\textbf{Theorem}}
\newtheorem*{lemma}{\textbf{Lemma}}
\newtheorem*{corollary}{\textbf{Corollary}}
\newtheorem*{proposition}{\textbf{Proposition}}
\newtheorem*{claim}{\textbf{Claim}}
\newtheorem*{conjecture}{\textbf{Conjecture}}

\theoremstyle{definition}
\newtheorem*{rk}{\textbf{Remark}}

\newcommand{\Summ}[1]{\underset{#1}{\sum}}
\newcommand{\sti}[2]{\left\{\begin{array}{c} #1 \\ #2 \end{array}\right\}}

\newcommand{\diam}{\emph{diam}}
\newcommand{\conv}{\mbox{Conv}}
\newcommand{\C}{\mathcal {C}}
\newcommand{\R}{\mathbb{R}}
\newcommand{\Z}{\mathbb{Z}}
\newcommand{\N}{\mathbb{N}}
\newcommand{\F}{\mathbb{F}}

\newcommand{\B}{\mathcal{B}}
\newcommand{\A}{\mathcal{A}}
\newcommand{\G}{\mathcal{G}}
\newcommand{\D}{\mathcal{D}}

\newcommand{\bpp}{\textbf{BPP}}
\newcommand{\np}{\textbf{NP}}
\newcommand{\p}{\textbf{P}}
\newcommand{\nl}{\textbf{NL}}
\newcommand{\conp}{\textbf{co\,-NP}}
\newcommand{\ph}{\textbf{PH}}
\newcommand{\dtime}{\textbf{DTIME}}
\newcommand{\sol}{\textbf{Solution:\,}}

\newcommand{\ov}[1]{\overline{#1}}

\newcommand{\nn}{\nonumber}

\thispagestyle{empty}

\usepackage{enumitem}

\usepackage{parskip}


\begin{document}

    {\Large MIRI, MAMME, MPAL -- Computational Complexity -- Homework 2}

    \vspace{0.5cm}

    \hrule

    \vspace{0.5cm}

    \begin{enumerate}[label=\textbf{Exercise \arabic*:}, leftmargin=0cm, labelwidth=-0.2cm, align=left]

        \item
            Prove that $\bpp^{\bpp} = \bpp$.

            \sol The inclusion $\bpp \subseteq \bpp^{\bpp}$ is clear.
            Let $L \in \bpp^{\bpp}$ and let $M$ be a probabilistic Turing machine
            that decides if $x \in L$
            in time $p(|x|)$, with error probability at most $\alpha$
            (both if $x \in L$ and if not);
            with access to an oracle for
            $A \in \bpp$.
            Furthermore, let $N$ be a probabilistic Turing machine that decides if $y \in A$
            in time $q(|y|)$,
            with error probability at most $\beta$
            (both if $y \in A$ and if not).

            Let $M'$ be the probabilistic Turing machine that simulates $M$,
            but instead of querying the oracle, it simulates $N$ on the input
            to the oracle.
            On an input $x$,
            The number of simulated queries is at most $p(|x|)$
            (one for each step of $M$).
            Because $M$ must write each query to the oracle,
            the inputs to $N$ are of length at most $p(|x|)$.
            Therefore, the total time of $M'$ is bounded by
            \[
               q(p(|x|)) \cdot p(|x|),
            \]
            which is polynomial in $|x|$.
            The probability of $M'$ giving the correct answer is
            \begin{align*}
                \gamma \geq & \mathbb{P}(\text{all simulated queries give the correct answer and } M' \text{ gives the correct answer}) \\
                = & \mathbb{P}(\text{all simulated queries are correct}) \cdot \mathbb{P}(M' \text{ gives the correct answer } | \text{ all simulated queries are correct}) \\
                \geq & \mathbb{P}(\text{all simulated queries are correct}) \cdot (1 - \alpha) \\
                \geq & (1 - \beta)^{p(|x|)} \cdot (1 - \alpha).
            \end{align*}
            The two last inequalities come from the fact that the randomness in the computational path
            of the simulated $M$ and the simulated oracle calls are all independent, and that
            $M'$ perfectly simmulates $M$ whenever all the simulated oracle calls give the correct answer.

            We have seen in class that the error rates of probabilistic Turing machines
            can be made arbitrarily small
            while keeping the runtime of the machines polynomial.
            In fact, we saw that they can be made less than
            \[
                \frac{1}{2^{t(|x|)}}
            \]
            for any polynomial $t$ and input $x$.
            For us to prove that $L \in \bpp$, it suffices to show that $\gamma \geq \frac{3}{4}$.
            For example, selecting
            \[
                \alpha \leq \frac{1}{2^3} = \frac{1}{8}
            \]
            and 
            \[
                \beta \leq \frac{1}{2^{10 p(|x|)}} \leq \frac{1}{10 p(|x|)},
            \]
            we obtain
            \[
                \gamma
                \geq \left(1 - \frac{1}{10 p(|x|)}\right)^{p(|x|)} \left( 1 - \frac{1}{8} \right)
                \geq \frac{9}{10} \cdot \frac{7}{8}
                > \frac{3}{4},
            \]
            where the middle inequality comes from the fact that the function
            \[
                f(n) = \left(1 - \frac{1}{10 n}\right)^{n}
            \]
            is increasing and $f(1) = \frac{9}{10}$.
        


    \end{enumerate}

\end{document}
