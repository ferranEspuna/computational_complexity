\documentclass{amsart}

\usepackage{amsthm}
\usepackage{amsmath}
\usepackage{amsfonts}
\usepackage{amssymb}
\usepackage{fullpage}
\usepackage{xcolor}
\usepackage{textcomp}
\usepackage{graphicx}
\usepackage{mathtools}
\usepackage{algorithm}
\usepackage{algpseudocode}
\newtheorem*{theirtheorem}{Theorem}
\newtheorem*{theirproposition}{Proposition}


\theoremstyle{plain}
\newtheorem*{theorem}{\textbf{Theorem}}
\newtheorem*{lemma}{\textbf{Lemma}}
\newtheorem*{corollary}{\textbf{Corollary}}
\newtheorem*{proposition}{\textbf{Proposition}}
\newtheorem*{claim}{\textbf{Claim}}
\newtheorem*{conjecture}{\textbf{Conjecture}}

\theoremstyle{definition}
\newtheorem*{rk}{\textbf{Remark}}

\newcommand{\Summ}[1]{\underset{#1}{\sum}}
\newcommand{\sti}[2]{\left\{\begin{array}{c} #1 \\ #2 \end{array}\right\}}

\newcommand{\diam}{\emph{diam}}
\newcommand{\conv}{\mbox{Conv}}
\newcommand{\C}{\mathcal {C}}
\newcommand{\R}{\mathbb{R}}
\newcommand{\Z}{\mathbb{Z}}
\newcommand{\N}{\mathbb{N}}
\newcommand{\F}{\mathbb{F}}

\newcommand{\B}{\mathcal{B}}
\newcommand{\A}{\mathcal{A}}
\newcommand{\G}{\mathcal{G}}
\newcommand{\D}{\mathcal{D}}

\newcommand{\size}[1]{\textbf{SIZE}(#1)}
\newcommand{\poly}{\text{poly}}
\newcommand{\advice}[2]{#1 | #2}
\newcommand{\bpp}{\textbf{BPP}}
\newcommand{\np}{\textbf{NP}}
\newcommand{\p}{\textbf{P}}
\newcommand{\nl}{\textbf{NL}}
\newcommand{\conp}{\textbf{co\,-NP}}
\newcommand{\ph}{\textbf{PH}}
\newcommand{\dtime}{\textbf{DTIME}}
\newcommand{\sol}{\textbf{Solution:\,}}

\newcommand{\ov}[1]{\overline{#1}}

\newcommand{\nn}{\nonumber}

\thispagestyle{empty}

\usepackage{enumitem}

\usepackage{parskip}


\begin{document}

    {\Large MIRI, MAMME, MPAL -- Computational Complexity -- Homework 2}

    \vspace{0.5cm}

    \hrule

    \vspace{0.5cm}

    \begin{enumerate}[label=\textbf{Exercise \arabic*:}, leftmargin=0cm, labelwidth=-0.2cm, align=left]

        \item Prove that $\bpp^{\bpp} = \bpp$.

            \sol The inclusion $\bpp \subseteq \bpp^{\bpp}$ is clear.
            Let $L \in \bpp^{\bpp}$ and let $M$ be a probabilistic Turing machine
            that decides if $x \in L$
            in time $p(|x|)$, with error probability at most $\alpha$
            (both if $x \in L$ and if not);
            with access to an oracle for
            $A \in \bpp$.
            Furthermore, let $N$ be a probabilistic Turing machine that decides if $y \in A$
            in time $q(|y|)$,
            with error probability at most $\beta$
            (both if $y \in A$ and if not).

            Let $M'$ be the probabilistic Turing machine that simulates $M$,
            but instead of querying the oracle, it simulates $N$ on the input
            to the oracle.
            On an input $x$,
            The number of simulated queries is at most $p(|x|)$
            (one for each step of $M$).
            Because $M$ must write each query to the oracle,
            the inputs to $N$ are of length at most $p(|x|)$.
            Therefore, the total time of $M'$ is bounded by
            \[
               q(p(|x|)) \cdot p(|x|),
            \]
            which is polynomial in $|x|$.
            The probability of $M'$ giving the correct answer is
            \begin{align*}
                \gamma \geq & \mathbb{P}(\text{all simulated queries give the correct answer and } M' \text{ gives the correct answer}) \\
                = & \mathbb{P}(\text{all simulated queries are correct}) \cdot \mathbb{P}(M' \text{ gives the correct answer } | \text{ all simulated queries are correct}) \\
                \geq & \mathbb{P}(\text{all simulated queries are correct}) \cdot (1 - \alpha) \\
                \geq & (1 - \beta)^{p(|x|)} \cdot (1 - \alpha).
            \end{align*}
            The two last inequalities come from the fact that the randomness in the computational path
            of the simulated $M$ and the simulated oracle calls are all independent, and that
            $M'$ perfectly simulates $M$ whenever all the simulated oracle calls give the correct answer.

            We have seen in class that the error rates of probabilistic Turing machines
            can be made arbitrarily small
            while keeping the runtime of the machines polynomial.
            In fact, we saw that they can be made less than
            \[
                \frac{1}{2^{t(|x|)}}
            \]
            for any polynomial $t$ and input $x$.
            For us to prove that $L \in \bpp$, it suffices to show that $\gamma \geq \frac{3}{4}$.
            For example, setting
            \[
                \alpha \leq \frac{1}{2^3} = \frac{1}{8}
            \]
            and
            \[
                \beta \leq \frac{1}{2^{10 p(|x|)}} \leq \frac{1}{10 p(|x|)},
            \]
            we obtain
            \[
                \gamma
                \geq \left(1 - \frac{1}{10 p(|x|)}\right)^{p(|x|)} \left( 1 - \frac{1}{8} \right)
                \geq \frac{9}{10} \cdot \frac{7}{8}
                > \frac{3}{4},
            \]
            where the middle inequality comes from the fact that the function
            \[
                f(n) = \left(1 - \frac{1}{10 n}\right)^{n}
            \]
            is increasing and $f(1) = \frac{9}{10}$.

        \newpage \item Describe the original definition of the advice class $\advice{\p}{\poly}$
        and give a high level explanation of why it is equivalent to the class $\size{\poly}$.

        \sol The advice classes $\advice{C}{A}$ are defined~\cite{karp1982turing} as the set of languages $L$
        for which there exists a language $L' \in C$ and a
        function $f_L: \N \to \{0, 1\}^*$ such that the function
        $n \mapsto |f_L(n)|$ is in $A$ (i.e., for $A=\poly$, $|f_L(n)|$ is polynomially bounded in $n$)
        satisfying
        \[
            L = \{x \in \{0, 1\}^* \mid (x, f_L(|x|)) \in L'\}.
        \]
        Intuitively, the language $L$ becomes a problem in $C$ by
        giving it some advice $f_L(|x|)$, which depends only on the length of the input $x$ and whose length $|f_L(|x|)|$ is bounded by a function in $A$.
        This is reminiscent of the definition of $\size{\mathcal{C}}$, in which the logic gates can depend on the input size.
        The definition, while perhaps intricate, precisely defines $L$ based on $L' \in C$ and the pre-determined advice $f_L(|x|)$.
        For any input $x$, its membership in $L$ is determined by whether
        $(x, f_L(|x|))$ is in $L'$; the advice $f_L(|x|)$ is an integral part of this decision process for each input length.

        In our case, we have $C = \p$ and $A = \poly$, which means that the language $L$
        can be decided by a polynomial-time Turing machine given some advice $f_L$
        of length bounded by a polynomial in the length of the input.
        In this case, the classes $\advice{\p}{\poly}$ and $\size{\poly}$ coincide.
        Let us sketch the proof of this by double inclusion.

        The inclusion $\advice{\p}{\poly} \subseteq \size{\poly}$ is similar to the one we saw in class,
        where we showed that
        \[
            \p \subseteq \size{\poly}
        \]
        by a reduction from a turing machine with a single tape, with extra states to simulate the read/write head,
        where the transformation was local in the sense that the sub-circuits representing each position of the tape
        were only connected to the sub-circuits representing the next and previous positions.
        In that case, the circuit just became bigger for each input length, as the number of positions we needed to simulate
        grew polynomially with the input length.
        Let $n \in \N$ be a fixed input length.
        To simulate the advice string $f_L(n)$, we ``hard-code''
        the input positions of the advice string into the circuit $S'$ corresponding to $L'$ for input size $n + |f_L(n)|$.
        These positions will behave just like the input positions of the tape but will no longer be connected to the actual input.
        We can do this because the advice function only depends on the length of the input, and not on the input itself.
        One way to hard-code the advice string into the circuit is to, for example, let $p$ be the first position of the actual input,
        and codify each position corresponding to the $i$th bit of $f_L(n)$ (that is, the ${(n+i)}$th
        position of the simulated input where $1 \leq i \leq |f_L(n)|$) as
        \begin{alignat*}{2}
            p &\text{ OR }  &&(\neg p), \text{ if } f_L(n)_i = 1 \\
            p &\text{ AND } &&(\neg p), \text{ if } f_L(n)_i = 0
        \end{alignat*}
        The number of gates in $S'$ (the circuit for $L'$) is a polynomial in $n + |f_L(n)|$
        (the total length of the actual input $x$ and the advice string $f_L(n)$),
        which in turn is bounded by a polynomial in $n$ because $|f_L(n)|$ (the length of the advice)
        is polynomially bounded in $n$.
        The number of extra gates needed to do the hard-coding of $f_L(n)$ is linear in $|f_L(n)|$, as we codify each bit independently.
        Therefore, we obtain a circuit $S$ whose size is bounded by a polynomial in $n$ and
        outputs $1$ for an input $x$ of length $n$ if and only if
        \[
            (x, f_L(n)) \in L' \iff x \in L.
        \]

        Now we show the other inclusion, $\size{\poly} \subseteq \advice{\p}{\poly}$.
        Let $L \in \size{\poly}$.
        This means that for each input length $n$, there exists a circuit $S_n$
        that decides $L$ for all inputs of length $n$, and the number of gates of $S_n$
        (denoted $|S_n|$) is bounded by a polynomial in $n$.
        We want to find a polynomial-time Turing machine $M$ and
        an advice string family $f_L(n)$ whose length is polynomially bounded in $n$,
        such that  for all inputs $x$ of length $n$,
        $M(x, f_L(n))$ accepts if and only if $x \in L$, that is, if $S_n$ accepts $x$.
        The idea is to encode a description of the circuit $S_n$ into the advice string
        $f_L(n)$ (which can be done because $|S_n|$ is polynomially bounded in $n$),
        and then let $M$ simulate $S_n$ on the input $x$.
        We need to find a succinct description $\langle S_n \rangle$ of $S_n$
        and describe a polynomial-time Turing machine $M$ that simulates $S_n$ given the input $x$ and $f_L(n) = \langle S_n \rangle$.

        We assume gates are topologically sorted $g_1, g_2, \dots, g_{|S_n|}$, meaning if gate $g_k$ inputs to $g_j$, then $k < j$.
        The first $n$ gates, $g_1, \dots, g_n$, correspond to the input bits $x_1, \dots, x_n$.
        Gates $g_{n+1}, \dots, g_{|S_n|}$ are internal logic gates.
        One gate (e.g., $g_{|S_n|}$) is the output gate.

        The advice string $f_L(n) = \langle S_n \rangle$ can specify:
        \begin{enumerate}
            \item \textbf{Total number of gates:} The value $|S_n|$, encoded in binary ($\mathcal{O}(\log |S_n|)$ bits).
            \item \textbf{Gate Information:} For each logic gate $g_j$ ($j \in \{n+1, \dots, |S_n|\}$):
            \begin{itemize}
                \item Its type (e.g., integer code for INPUT, AND, OR, NOT; $\mathcal{O}(1)$ bits).
                \item Indices of its input gate(s) (e.g., $(k_1, k_2)$ for binary gates, $k_1$ and a dummy value for unary, where $k_1, k_2 < j$).
                Each index takes $\mathcal{O}(\log |S_n|)$ bits.
            \end{itemize}
        \end{enumerate}
        The total length of $\langle S_n \rangle$ is $\mathcal{O}(|S_n| \log |S_n|)$.
        Since $|S_n| \in \poly(n)$, $\log |S_n| = \mathcal{O}(\log n)$,
        so the length of $\langle S_n \rangle$ is $\mathcal{O}(\poly(n) \log n)$, which is polynomially bounded in $n$.

        The language $L'$ is defined as
        \[
            \textbf{SIM}
            = \{ (x, \langle S \rangle) \mid S \text{ has } |x| \text{ inputs and accepts } x\}.
        \]
        Therefore,
        \[
            L = \{ x \in \{0, 1\}^* \mid S_{|x|} \text{ accepts } x \}
            = \{ x \in \{0, 1\}^* \mid (x, f_L(|x|)) \in \textbf{SIM} \}.
        \]
        If we show that $\textbf{SIM} \in \p$, we will have shown that $L \in \advice{\p}{\poly}$.
        We now provide pseudocode for the Turing machine $M_{\textbf{SIM}}$ that simulates $S$ on $x$ given $(x, \langle S \rangle)$.

        \begin{algorithm}[H] % H makes the algorithm float "here" if possible
        \caption{Circuit Simulation by $M_{\textbf{SIM}}$}
        \label{alg:circuitsim_m}
        \begin{algorithmic}[1]
        \Procedure{SimulateCircuit}{$x$, $\langle S_n \rangle$}
            \State $N_{gates}, GATES \gets \langle S_n \rangle$ \Comment{Parse the circuit description into a list of gates}
            \State $n \leftarrow |x|$

            \State Declare $GateValues[1 \dots N_{gates}]$ \Comment{Initialize an array to store the values of all gates}

            \For{$i \leftarrow 1$ to $n$}
                \State $GateValues[i] \leftarrow x_i$ \Comment{$x_i$ is the $i$-th bit of $x$}
            \EndFor

            \For{$j \leftarrow n+1$ to $N_{gates}$}
                \State $(t, k_1, k_2) \gets GATES[j]$ \Comment{Get the type and input indices of gate $j$}
                \If{$t$ is NOT}
                    \State $GateValues[j] \gets \neg GateValues[k_1]$
                \ElsIf{$t$ is AND}
                    \State $GateValues[j] \gets GateValues[k_1] \land GateValues[k_2]$
                \ElsIf{$t$ is OR}
                    \State $GateValues[j] \gets GateValues[k_1] \lor GateValues[k_2]$
                \EndIf
            \EndFor

            \State \Return $GateValues[N_{gates}]$
        \EndProcedure
        \end{algorithmic}
        \end{algorithm}

        The algorithm clearly runs in polynomial time,
        so we have shown that $\textbf{SIM} \in \p$, concluding our proof that
        \[
            \size{\poly} \subseteq \advice{\p}{\poly}.
        \]
    \end{enumerate}

    \newpage
    \bibliography{main}
    \bibliographystyle{plain}

\end{document}
