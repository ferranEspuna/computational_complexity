\documentclass{amsart}

\usepackage{amsthm}
\usepackage{amsmath}
\usepackage{amsfonts}
\usepackage{amssymb}
\usepackage{fullpage}
\usepackage{xcolor}
\usepackage{textcomp}
\usepackage{graphicx}
\usepackage{mathtools}
\usepackage{algorithm}
\usepackage{algpseudocode}
\newtheorem*{theirtheorem}{Theorem}
\newtheorem*{theirproposition}{Proposition}


\theoremstyle{plain}
\newtheorem*{theorem}{\textbf{Theorem}}
\newtheorem*{lemma}{\textbf{Lemma}}
\newtheorem*{corollary}{\textbf{Corollary}}
\newtheorem*{proposition}{\textbf{Proposition}}
\newtheorem*{claim}{\textbf{Claim}}
\newtheorem*{conjecture}{\textbf{Conjecture}}

\theoremstyle{definition}
\newtheorem*{rk}{\textbf{Remark}}

\newcommand{\Summ}[1]{\underset{#1}{\sum}}
\newcommand{\sti}[2]{\left\{\begin{array}{c} #1 \\ #2 \end{array}\right\}}

\newcommand{\diam}{\emph{diam}}
\newcommand{\conv}{\mbox{Conv}}
\newcommand{\C}{\mathcal {C}}
\newcommand{\R}{\mathbb{R}}
\newcommand{\Z}{\mathbb{Z}}
\newcommand{\N}{\mathbb{N}}
\newcommand{\F}{\mathbb{F}}

\newcommand{\B}{\mathcal{B}}
\newcommand{\A}{\mathcal{A}}
\newcommand{\G}{\mathcal{G}}
\newcommand{\D}{\mathcal{D}}

\newcommand{\size}[1]{\textbf{SIZE}(#1)}
\newcommand{\poly}{\text{poly}}
\newcommand{\advice}[2]{#1 | #2}
\newcommand{\bpp}{\textbf{BPP}}
\newcommand{\np}{\textbf{NP}}
\newcommand{\p}{\textbf{P}}
\newcommand{\nl}{\textbf{NL}}
\newcommand{\conp}{\textbf{co\,-NP}}
\newcommand{\ph}{\textbf{PH}}
\newcommand{\fp}{\textbf{FP}}
\newcommand{\sharpp}{\textbf{\#P}}
\newcommand{\sharpclique}{\textbf{\#CLIQUE}}
\newcommand{\sharpsat}{\textbf{\#SAT}}
\newcommand{\dtime}{\textbf{DTIME}}
\newcommand{\sol}{\textbf{Solution:\,}}

\newcommand{\ov}[1]{\overline{#1}}

\newcommand{\nn}{\nonumber}

\thispagestyle{empty}

\usepackage{enumitem}

\usepackage{parskip}


\begin{document}

    {\Large MIRI, MAMME, MPAL -- Computational Complexity -- Homework 4}

    \vspace{0.5cm}

    \hrule

    \vspace{0.5cm}

    \begin{enumerate}[label=\textbf{Exercise \arabic*:}, leftmargin=0cm, labelwidth=-0.2cm, align=left]

        \item Let \sharpclique\, be the problem of counting how many $k$-cliques exist in a given graph
        for a given positive integer $k$.
        Show that $\sharpclique \in \p^{\sharpsat}$.

           \sol I proceed by direct reduction.
            Let $G = (V, E)$ be an undirected graph with $n$ vertices
            and $1 \leq k \leq n$ an integer.
            I now claim that the number $N'$ of \emph{ordered}
            $k$-cliques can be found in polynomial time
            with access to a $\sharpsat$ oracle.
            This is enough because given the number
            of ordered $k$-cliques is $N' = k!N$ and
            $N' \mapsto N'/k! \in \fp$.

            Let $T = (u_1, \dots u_k) \in V^k$ be a $k$-tuple of vertices.
            The elements of $T$ form a $k$-clique in $G$ if and only if
            $\{u_i, u_j\} \in E$ for all $1 \leq i < j \leq k$
            (note that this implies that all the elements are different,
            since I am assuming that $G$ is a simple graph, with no loops).
            To encode this information into a boolean formula I consider
            the variables $x_i^u$ (for $u \in V$ and $i \in [k])$,
            representing whether $u_i = u$.
            Define
            \[
                F \coloneqq \{(u, w) \in V^2 \mid \{u, w\} \notin E\}.
            \]
            Note that this includes the diagonal entries $(u, u)$.
            The clique condition can then be expressed as
            \[
                  \mathcal{C} \coloneqq \bigwedge_{(u, w) \in F, (i, j) \in \binom{[k]}{2}} (\neg x_i^u \lor \neg x_j^w),
            \]
            which is in CNF.
            An assignment $V \times [k] \to \{0, 1\}$ represents a $k$-tuple of vertices if and only if
            exactly one vertex is selected for each entry of $T$.
            I do this in two steps.
            First I introduce a formula that ensures \emph{at most} one vertex is selected for each entry:
            \[
                \mathcal{B} \coloneqq \bigwedge_{\{u, w\} \in \binom{V}{2},\, i \in [k]} (\neg x_i^u \lor \neg x_i^w).
            \]
            Then, I introduce the following other formula, which ensures \emph{at least} one vertex is selected
            for each entry.
            \[
                \mathcal{A} \coloneqq \bigwedge_{i \in [k]} \left( \bigvee_{u \in V} x_i^u \right).
            \]
            Formulas $\mathcal{A}, \mathcal{B}$ and $\mathcal{C}$
            can clearly be obtained from $G$ (represented, for example, as an adjacency matrix)
            in polynomial time.
            Furthermore, there is a bijection between assignments
            $V \times [k] \to \{0, 1\}$ satisfying
            $\mathcal{A} \land \mathcal{B} \land \mathcal{C}$
            and ordered $k$-cliques in $G$.
            Therefore, a single query to a $\sharpsat$ oracle is enough to count the number $N'$ of ordered
            $k$-cliques in $G$ in polynomial time.
            Then computing $N = N' / k!$ yields the actual number of $k$-cliques.

        \newpage
        \item We know that $\ph \subseteq \p^{\sharpp}$ (Toda's theorem).
        What would be the consequences of the reverse inclusion $\p^{\sharpp} \subseteq \ph$?

        \sol Suppose that $\p^{\sharpp} \subseteq \ph$.
        I will argue that the polynomial hierarchy collapses to a finite level.
        consider the language
        \[
            L = \{ \left< \mathcal{F}, k \right> \mid \mathcal{F} \text{ is a formula in CNF with less than } k \text{ solutions}\}.
        \]
        Clearly, $L \in \p^{\sharpp}$, since given $\left< \mathcal{F}, k \right>$ one
        can count the number of solutions to $\mathcal{F}$ with a $\sharpsat$ oracle
        call and then check whether this number is less than $k$, all in polynomial time.
        By assumption, $L \in \ph$ and in particular $L \in \mathbf{\Sigma}^p_i$ for some $i \geq 0$.
        Consider now another language $A \in \ph$.
        By Toda's theorem, $A \in \p^{\sharpp} = \p^{\sharpsat}$.
        Suppose that a machine $M$ decides $A$ in polynomial time with access to a $\sharpsat$ oracle.
        I now argue that each call to the $\sharpsat$ oracle can be replaced by a
        polynomial-time computation with access to an oracle for $L$,
        obtaining a machine $M'$ that decides $A$ in polynomial time with access to an oracle for $L$.
        Therefore, $A \in \p^{L} \subseteq \p^{\mathbf{\Sigma}^p_i} = \mathbf{\Delta}^p_{i+1}$, proving that $\ph = \mathbf{\Delta}^p_{i+1}$.

        Let us now see the promised reduction.
        Suppose the computation of $M$ on input $x$ takes $p(|x|)$ steps,
        where $p$ is a polynomial.
        The number of variables in the formula $\mathcal{F}$ of an oracle call
        in such a computation is at most $|\mathcal{F}| \leq |x| + p(|x|) = p'(|x|)$,
        since the formula has to be written on the tape of the machine.
        The number of solutions to $\mathcal{F}$ is at most $2^{p'(|x|)}$.
        The exact number of solutions to $\mathcal{F}$ can be computed
        by running binary search on the range $[0, 2^{p'(|x|)}]$,
        taking $\mathcal{O}(\log(2^{p'(|x|)})) = \mathcal{O}(p'(|x|))$ iterations
        and making a call to the oracle for $L$ in each iteration.
        The number of steps of each iteration is in the order of the size of the integers that are being compared and operated on,
        which is also $\mathcal{O}\left(\log(2^{p'(|x|)})\right) = \mathcal{O}\left(p'(|x|)\right)$.
        All in all, the call to the oracle for $\sharpsat$ can be replaced by a computation
        with access to an oracle for $L$ that takes $\mathcal{O}\left(p'(|x|)^2\right)$ steps,
        and $M'$ runs in time $\mathcal{O} \left( p(|x|) p'(|x|)^2 \right)$,
        justifying the claim that $A \in \p^{L}$.

    \end{enumerate}

\end{document}
